\documentclass[12pt,twoside,openright,a4paper]{article}
%\documentclass[12pt,twoside,openright,usletter]{article}
% !TeX spellcheck = en_US
%\documentclass[11pt]{article}
% UK date format in bibliography:
\usepackage[british]{babel}
\usepackage[inner=25mm,outer=25mm,top=20mm,bottom=20mm]{geometry}

%\usepackage[UKenglish]{isodate}%UK date endian
\usepackage[headings]{fullpage}
\usepackage[hidelinks]{hyperref}

% Bibliography:
\usepackage[utf8]{inputenc}
\usepackage{csquotes,xpatch}% recommended
% list up to 99 names instead of the default 3
\usepackage[backend=biber,bibencoding=utf8,style=numeric,maxnames=99,backref=false,sortcites,datamodel=thesis]{biblatex}
\addbibresource{cheri.bib}
\AtEveryBibitem{%
% Don't print ISBN,issn, or URL dates
\clearfield{issn}%
\clearfield{isbn}%
\clearfield{urldate}%
\clearfield{urlyear}%
}

\usepackage{bytefield}
\usepackage{color}
\usepackage[scaled=0.8]{DejaVuSansMono}
\usepackage[T1]{fontenc}
\usepackage{listings}
\usepackage{mdframed} % To avoid linebreaks in lstlistings
\lstnewenvironment{clisting}[1][]{\endgraf\noindent\minipage{\linewidth}\lstset{language={C},breaklines=true,frame=L,#1}}{\endminipage\endgraf}
\lstnewenvironment{compilerwarning}[1][]{\endgraf\noindent\minipage{\linewidth}\lstset{language={},breaklines=true,basicstyle=\scriptsize\ttfamily\bfseries,frame=L,#1}}{\endminipage\endgraf}

\usepackage{subcaption}
\usepackage{times}
\usepackage{url}
\usepackage[svgnames]{xcolor}
\definecolor{lightgray}{gray}{0.8}
\usepackage{xspace}
\usepackage{xfrac}

\usepackage[nameinlink,noabbrev,capitalise]{cleveref}

% drawing over lstlistings (code stolen from nwf)
\usepackage{tikz}
   \usetikzlibrary{decorations.pathreplacing}
   \usetikzlibrary{fit}
   \usetikzlibrary{tikzmark}
   \usetikzlibrary{calc}
   \usetikzlibrary{patterns}
\newcommand*{\vcpgfmark}[1]{\ensuremath{\vcenter{\hbox{\pgfmark{#1}}}}}
% GBP symbol should be safe since it's easy to enter (at least on a UK keyboard) and won't be in any valid lstlistings
\lstset{escapechar=£} % Note: ensure this doesn't occur in any of the code
\newcommand{\TikzListingHighlight}[3][]{\tikz[overlay,remember picture]{\draw[\ifstrempty{#1}{yellow}{#1}, line width=10pt,opacity=0.3](#2) -- (#3);}}
\newcommand*{\TikzListingHighlightStartEnd}[2][]{\tikz[overlay,remember picture]{\draw[\ifstrempty{#1}{yellow}{#1}, line width=10pt,opacity=0.3](pic cs:Start#2) -- (pic cs:End#2);}}


\renewcommand{\UrlFont}{\ttfamily\small}

\newcommand{\baselineboxformatting}[1]{%
  % Measure size of contents
  \sbox0{#1}%
  % Use the difference between the contents' height and the bitbox's height,
  % clamped to [-.44\baselineskip, 0], as our minimum depth.
  \setlength{\skip0}{\ht0 - \height}%
  \ifdim\skip0>0pt%
    \setlength{\skip0}{0}%
  \else%
    \ifdim\skip0<-.44\baselineskip%
      \setlength{\skip0}{-.44\baselineskip}%
    \fi%
  \fi%
  \centering\rule[\skip0]{0pt}{\height}#1}
\bytefieldsetup{boxformatting=\baselineboxformatting}

\lstset{basicstyle=\footnotesize\ttfamily}
%\newcommand{\ccode}[1]{\lstinline[language={C}]{#1}}
%\newcommand{\cxxcode}[1]{\lstinline[language={C++}]{#1}}
\newcommand{\ccode}[1]{{\small\ttfamily{#1}}}
\newcommand{\cxxcode}[1]{{\ccode{#1}}}
\newcommand{\cconst}[1]{{\ccode{#1}}}
\newcommand{\cfunc}[1]{{\ccode{#1()}}}
\newcommand{\cvar}[1]{{\ccode{#1}}}
\newcommand{\pathname}[1]{{\ccode{#1}}}
\newcommand{\commandline}[1]{{\ccode{#1}}}

\newcommand{\ptrdifft}{{\ccode{ptrdiff\_t}}\xspace}
\newcommand{\maxalignt}{{\ccode{max\_align\_t}}\xspace}
\newcommand{\sizet}{{\ccode{size\_t}}\xspace}
\newcommand{\ssizet}{{\ccode{ssize\_t}}\xspace}
\newcommand{\ptraddrt}{{\ccode{ptraddr\_t}}\xspace}
\newcommand{\cuintptrt}{{\ccode{uintptr\_t}}\xspace}
\newcommand{\cintptrt}{{\ccode{intptr\_t}}\xspace}
\newcommand{\ccharstar}{{\ccode{char *}}\xspace}
\newcommand{\cvoidstar}{{\ccode{void *}}\xspace}
\newcommand{\clongt}{{\ccode{long}}\xspace}
\newcommand{\cintt}{{\ccode{int}}\xspace}
\newcommand{\cintttt}{{\ccode{int32\_t}}\xspace}
\newcommand{\cintsft}{{\ccode{int64\_t}}\xspace}
\newcommand{\cintcapt}{{\ccode{intcap\_t}}\xspace}
\newcommand{\cuintcapt}{{\ccode{uintcap\_t}}\xspace}

\newcommand{\uucap}{{\ccode{\_\_capability}}\xspace}

\newcommand{\SIGPROT}{{\ccode{SIGPROT}}\xspace}

\newcommand{\futurevariant}[1]{{\color{teal} #1}}
\newcommand{\morellovariant}[1]{{\color{red} #1}}

\newcommand{\note}[2]{{\color{blue}[ Note: #1 - #2]}}
\usepackage{xstring}
\IfSubStr*{\jobname}{final}{
  \renewcommand{\note}[2]{\relax\ifhmode\unskip\fi}
}{
  % show comments by default
}

\newcommand{\arnote}[1]{\note{#1}{Alex R.}}
\newcommand{\bdnote}[1]{\note{#1}{Brooks D.}}
\newcommand{\rwnote}[1]{\note{#1}{Robert W.}}
\newcommand{\amnote}[1]{\note{#1}{Alfredo M.}}
\newcommand{\psnote}[1]{\note{#1}{Peter S.}}
\newcommand{\pgnnote}[1]{\note{#1}{Peter N.}}
\newcommand{\jrtcnote}[1]{\note{#1}{Jess C.}}
\newcommand{\hmnote}[1]{\note{#1}{Hesham A.}}
\newcommand{\nwfnote}[1]{\note{#1}{nwf}}
\newcommand{\dcnote}[1]{\note{#1}{David}}
\newcommand{\jhbnote}[1]{\note{#1}{John B.}}

% typeset C++ sensibly
% NB: \nolinebreak was only made robust upstream on 2019-08-20
\usepackage{relsize}
\newcommand*{\cpp}{\texorpdfstring{C\textsmaller[2]{\protect\nolinebreak[4]\hspace{-.05em}\raisebox{.45ex}{\textbf{++}}}}{C++}}
\newcommand*{\COrCpp}{C/\cpp{}}
\newcommand*{\purecapCOrCpp}{CHERI \COrCpp{}}
\newcommand*{\hybridCOrCpp}{Hybrid-Capability \COrCpp{}}
\newcommand*{\CHERIhybridCOrCpp}{CHERI \hybridCOrCpp{}}

\newcommand*{\CAndCpp}{C and \cpp{}}
\newcommand*{\hybridCAndCpp}{Hybrid-Capability \CAndCpp{}}
\newcommand*{\CHERIhybridCAndCpp}{CHERI \hybridCAndCpp{}}

\hyphenation{Free-BSD}
\hyphenation{Free-RTOS}
\hyphenation{Cheri-BSD}
\hyphenation{Cheri-Free-RTOS}
\hyphenation{Cheri-ABI}
\hyphenation{Web-Kit}
\hyphenation{Postgre-SQL}

\title{\CHERIhybridCOrCpp{} \\
  Programming Guide \\
  (DRAFT)}
\author{Robert~N.~M.~Watson$^*$,
    Alexander~Richardson$^*$,
    Jessica~Clarke$^*$,
    Brooks Davis$^\dagger$, \\
    John Baldwin$^\ddagger$,
    David Chisnall$^\S$,
    Nathaniel Filardo$^\S$,
    Simon W. Moore$^*$, \\
    Edward Napierala$^*$,
    Peter Sewell$^*$,
    Peter~G.~Neumann$^\dagger$
  \\
    \textbf{(others to be added)}
  \\
  \\
  $^*$University of Cambridge, $^\dagger$SRI International, \\
  $^\ddagger$Ararat River Consulting, LLC and $^\S$Microsoft Research}
%
%Alexander Richardson$^*$,
%  Brooks Davis$^\dagger$, \\
%  John Baldwin$^\ddagger$, David Chisnall$^\S$, Jessica Clarke$^*$,
%  Nathaniel Filardo$^*$, \\
%  Simon W. Moore$^*$,  Edward Napierala$^*$, Peter Sewell$^*$, and \\
%  Peter G. Neumann$^\dagger$ \\
%  \\

\begin{document}
\sloppy

%% CL tech-report format provides its own cover page.  Comment for final
%% version.
\maketitle

%% CL tech-report format requires page numbering to start at 3.  Uncomment for
%% final version.
%\setcounter{page}{3}
%%

%
% Keep Abstract in sync with the Introduction.
%
\newcommand{\abstracttext}{

This document is an introduction to the \CHERIhybridCAndCpp{} programming
languages.
\hybridCOrCpp{} allows the selective use of CHERI capabilities from within an
otherwise unmodified, and Application Binary Interface (ABI)-compatible,
C/\cpp{}-language code base.
Its aim is to allow management of, and interoperation with, capability-enabled
code while using an integer rather than capability pointer implementation
internally.
Unlike pure-capability \purecapCOrCpp{}, \hybridCOrCpp{} defaults to an
integer implementation for pointers except where specifically annotated in the
program source code.
Due to the requirement for ABI compatibility, implied pointers within the
language runtime itself are also primarily implemented as integers rather than
capabilities.

\hybridCOrCpp{} primarily sees use in specialized low-level systems code that,
for compatibility or performance reasons, must use non-capability pointers, or
on the boundaries between pure-capability and conventional generated code.
Current use cases includes boot loaders, operating-system kernels, and
Inter-Process Communication (IPC) runtimes.
Potential future use cases include programs that have been formally verified
(which do not require dynamic memory safety) and code that has high static or
dynamic pointer density (leading to unaffordable memory-allocation footprints
or cache utilization and energy use).

\nwfnote{Do you want to call out partially adapted compartmentalized programs
(possibly inclusive of programs with their own runtime systems; one could
imagine enriching JS to have a ForeignCapability type, e.g.) as another
possible example where being able to exchange capabilities might be useful
despite possibly having a non-NULL DDC?}

%Hybrid code is used in the CheriBSD kernel, and in the userspace run-time
%environment for hybrid processes to enable communication with pure-capability
%processes via co-process IPC.

%Because the benefits of referential, spatial, and temporal memory safety are
%unavailable to both the program and language runtime, care should be utilized,
%and \hybridCOrCpp{} should be used only where strictly necessary.
Because the benefits of CHERI referential, spatial, and temporal memory safety are
unavailable to both the program and language runtime in \hybridCOrCpp{},
its use requires care and should be avoided unless strictly necessary.

}

\newcommand{\reviewwarning}{
\textbf{
As \hybridCOrCpp{} remains an area of active research and development, this
report is a request for review and comments rather than a specification.
}
\rwnote{This last sentence will go away in a final version.}
}

\begin{abstract}
\abstracttext

\reviewwarning
\end{abstract}

\newpage
\setcounter{tocdepth}{2}
\tableofcontents

\newpage

\section{Introduction}

%
% Keep Abstract in sync with the Introduction.
%
\abstracttext

This is a companion to the \textit{\purecapCOrCpp{} Programming
Guide}~\cite{UCAM-CL-TR-947}, and refers to that report rather than
replicating its contents.
Readers will benefit from having read that document prior to this one, as
many behaviors in \hybridCOrCpp{} are based on those in \purecapCOrCpp{}, or
will be contrasted with it in this guide.
The \textit{Introduction to CHERI} technical
report~\cite{UCAM-CL-TR-941} also provides background material for this
guide.

\subsection{Objective of \hybridCOrCpp{}}

The aim of \hybridCOrCpp{} is to allow programs to be written using
conventional integer pointers throughout their implementations, and yet also
to interact with capabilities via modest C-language extensions causing
specifically annotated pointers to be implemented using capabilities.
This is in contrast to \purecapCOrCpp{}~\cite{UCAM-CL-TR-947}, which
implements all pointers as capabilities by default, and whose aim is strong
referential, spatial, and temporal memory protection for C/\cpp{}.
\psnote{including ``temporal'' there, unqualified, is going to be confusing.  Suggest splitting the sentence up and describing more explicitly what the current state and plausible options are w.r.t. temporal}
While \hybridCOrCpp{} is unable to provide strong referential, spatial, or
temporal safety in most use cases, it allows conventional
integer-pointer-based code to interact usefully with capabilities in a number
of specific use cases.
\psnote{that seems to underplay what one can get from hybrid -- surely one can say something about ``specific protections'' or ``protection in specific ways'' or something?}
We explore existing code corpora for \hybridCOrCpp{} in further detail in
Section~\ref{section:hybrid-c-examples}.
These include the hybrid CheriBSD userspace and hybrid userspace examples.

\subsection{Example code}

\hybridCOrCpp{} source code appears identical to off-the-shelf C and \cpp{}
code, and for most intents and purposes behaves identically.
The principal difference is that annotation of pointer types with \uucap{}
causes pointers of those types to be implemented using architectural
capabilities, and that various \purecapCOrCpp{} APIs are available:

\begin{lstlisting}[language=C]
#include <cheriintrin.h>
#include <cheri.h>

/*
 * Function to convert an integer pointer into a capability pointer, setting
 * a specific length on it.  Bounds may be imprecise if the base alignment of
 * c is not suitable for the length, or the length is not suitably padded.
 */
void * __capability
capptr_from_intptr(void *c, size_t len)
{
	void * __capability cap;

	/*
	 * Create a capability pointer from an integer pointer.  This
	 * qualifier silences a compiler warning by documenting that this is
	 * an intentional cast between pointer implementations.
	 */
	cap = (__cheri_tocap void *)c;

	/*
	 * Restrict bounds to (cap, cap + len) -- assuming suitable alignment
	 * and padding.  The cheri_bounds_set_exact() API variant will return
	 * precise bounds or throw a hardware exception.  This variant
	 * produces "sloppy" bounds.
	 */
	cap = cheri_bounds_set(cap, len);

	return (cap);
}
\end{lstlisting}

\noindent
Complete documentation of the APIs for capability management can be found in
the \textit{\purecapCOrCpp{} Programming Guide}~\cite{UCAM-CL-TR-947}.
That document also describes properties such as imprecise bounds and the
implications for CHERI-aware memory allocators that require precise bounds for
spatial safety purposes.

\subsection{Document structure}

This report describes how \hybridCAndCpp{}:

\begin{itemize}
\item Interact with explicit pointers in program source code, as well as
  implied pointers in the language runtime and generated code;
\item Affect standard C-language types (such as \cintptrt{}), casts, and the
  address-of operator (\ccode{\&});
\item May support extended C and POSIX memory-management APIs;
\item Are used in our current larger code corpora, such as CheriBSD's
  CheriABI system-call layer~\cite{davis2019:cheriabi}; and
\item Suffer from a number of limitations causing us to recommend its use
  only in very specific use cases.
\end{itemize}

\noindent
We also recommend further reading in the form of several of our papers and
technical reports.

\subsection{Status of this document}

This document is a request for comments and feedback, and not a stable
specification.
This document currently describes three variants of \hybridCOrCpp{}:

\begin{enumerate}
\item \hybridCOrCpp{} as implemented in the CHERI Clang/LLVM implementation on
  CHERI-MIPS and CHERI-RISC-V.

  Unless otherwise indicated, all statements refer to this model.

\item \hybridCOrCpp{} as implemented in the CHERI Clang/LLVM implementation on
  Morello (which differs in small but important ways).

  Key differences are in \morellovariant{red text}.

\item Our current thinking on next directions for \hybridCOrCpp{}.

  Key differences are in \futurevariant{teal text}.
\end{enumerate}

\noindent
We have attempted to clearly comment on these behavioral divergences where
required.
It is our aim to achieve convergence of these approaches so that a single
\CHERIhybridCOrCpp{} can be used across all architectures, and best addresses
all use cases.
Feedback on this draft may be submitted via pull requests or the issue tracker
on our GitHub repository for the report:

\smallskip

\url{https://github.com/CTSRD-CHERI/cheri-hybrid-c-guide}

\section{Pointer implementation}

We are concerned with pointers arising from two parts of the \COrCpp{} implementation:

\begin{description}
\item[Explicit pointers] are pointers visible in the program source.
  These include declared pointers taken to local, global, or thread-local
  variables, to functions, or pointers to returned heap allocations.
  In \hybridCOrCpp{}, these pointers will be implemented using capabilities
  only if specifically annotated using a \uucap qualifier.
  When considering \cpp, we also include references in this category, as
  the language runtime generally implements them in the same way as pointers.

\item[Implied pointers] are those used by the language runtime and generated
  code to manage the language implementation itself.
  These implied pointers are required to implement language features such:

  \begin{itemize}
  \item The stack using stack pointers and frame pointers,
  \item Function invocation, function pointers, and function return using
    Program Linkage Tables (PLTs) and return addresses,
  \item Global variables using the Global Offset Table (GOT) or, for
    pure-capability code, capability table (captable),
  \item Thread-local storage using per-thread TLS pointers,
  \item \cpp{} method dispatch using vtable pointers and vtables, and
  \item The \ccode{this} pointer.
  \end{itemize}
\end{description}

\noindent
In \hybridCOrCpp{}, explicit and implied pointers are implemented using two underlying
architectural types:

\begin{description}
\item[Integer addresses:]
  Storage size and alignment will be the architectural address size (e.g.,
  64 bits on a 64-bit architecture in a 64-bit process environment).

  Generated code will utilize integer load and store instructions to access
  variables of these types, and dereference them using integer-relative
  instructions.

  Explicit pointers will be of this type by default.
  Implied pointers will be of this type if required by the existing ABI.

\item[CHERI capabilities:]
  Storage size and alignment will be the architectural capability size (e.g.,
  128 bits on a 64-bit architecture in a 64-bit process environment, with the
  tag bit for each capability-sized-and-aligned unit of memory maintained by
  the hardware in non-addressable memory).

  Generated code will utilize capability load and store instructions to access
  variables of these types, and dereference them using capability-relative
  instructions.

  Explicit pointers will only be of this type if specifically annotated in the
  program source.
  Implied pointers may be of this type if it does not harm the ABI; some
  implied pointers may be replicated in order to allow both types to be used
  (e.g., by providing a GOT and a captable).
\end{description}

\noindent
In the following sections, we consider the implementation of pointers in
C/\cpp{} source code and also the language runtime.

\arnote{There is still a lot of duplication here, will try to reword it more later.}

\subsection{Explicit pointers}

To implement an explicit pointer as a capability in \hybridCOrCpp{}, use the
qualifier \uucap{} on a pointer type.
The following code excerpt declares a capability pointer to a \ccode{char}:

\begin{lstlisting}[language=C]
char * __capability c;
\end{lstlisting}

\noindent
The \uucap qualifier can also be applied to \cpp{} references.
For example, the following code excerpt declares a capability reference to a
\ccode{char}:

\begin{lstlisting}[language=C]
void myclass::mymethod(char & __capability c)
\end{lstlisting}

\noindent
Unqualified explicit pointers and \cpp{} references will be implemented as
integer addresses.

\subsection{Implied pointers}

In conventional code generation, implied pointers are implemented using
integers.
In pure-capability \purecapCOrCpp{}, implied pointers  are implemented using
capabilities.
In \hybridCOrCpp{}, these pointers are generally implemented using integers,
although in some cases, underlying data structures can be replicated to
provide capability versions.
For example, it is possible to provide an integer GOT alongside a
capability-based captable, allowing global variable access to be achieved
via either mechanism depending on how code is generated.

We consider three cases:

\begin{description}
\item[Implied pointers critical to the ABI]
  Wherever the ABI dictates that implied pointers must be integer pointers,
  that will be maintained by \hybridCOrCpp{} code generation and in the
  language runtime.
  For example, the hybrid ABI requires that the stack pointer be an integer
  pointer.
  \dcnote{This isn't clear to me as a general point.  An ISA that had an
    architectural stack pointer could have a stack capability and define the
    legacy ops that write an integer to the stack pointer to set the address.
    We couldn't do that on MIPS because the stack pointer is non-architectural
    but it would be possible on ARM.}

\item[Implied pointers invisible to the ABI]
  In some cases, implied pointers may be implemented as capability pointers
  without impacting the ABI.

\psnote{again flag more clearly that this is speculative?  The blue colour alone isn't doing it for me}
  
  \futurevariant{For example, hybrid code could use a capability return address
  rather than an integer return address to provide pointer provenance validity
  for the return address, even if not tight bounds, as the return address is
  not part of the ABI to other functions.}

  \jrtcnote{\cfunc{\_\_builtin\_return\_address} does leak these implied
  pointers to some code, and it has implications for things like libunwind, so
  it does affect the ABI. Also the kernel needs to know to put a capability in
  CRA in order for signal handlers to be able to return to \cfunc{sigcode}.
  There's a lot of subtlety here that needs to be captured.}

  \rwnote{I'd been pondering whether the ABI might allow us to use a return
  capability, but in which unwind/etc only looked at the lower 64 bits...  And
  I suppose the builtin could be polymorphic.  But signals are indeed a messy
  issue.
  Is there a better example we could give?}

  \jhbnote{Not as true for MIPS, but on many architectures, the
    position of the saved PC and frame pointer in the stack is part
    of the ABI (if you compile with -fno-omit-frame-pointer as
    FreeBSD's kernel does, and as all of illumos does).  DDB in the
    kernel relies on this convention for stack unwinding even on
    non-x86 platforms like RISC-V.  DTrace also relies on it (so for
    working userspace dtrace FBT probes I think you have to compile
    everything with -fno-omit-frame-pointer).  My knee-jerk reaction
    is that the RA is really part of the ABI in practice.}

\item[Implied pointers that can be replicated as integer and capability
  pointers]
  In other cases, implied pointers may be replicated, with a full set of
  integer pointer variants provided for the existing ABI, but additional
  capability pointer variants provided for hybrid code.

  \futurevariant{For example, a separate captable could be provided,
  replicating the set of entries in the existing GOT, to allow capability
  pointers taken to global variables to have desired bounds and permissions.}

  \morellovariant{In the Morello ABI, capability pointers taken to globals
  pick up bounds and other metadata from constant pools of capabilities,
  effectively small per-function captables manually created by the compiler.}

  \psnote{really the first Morell-specific thing?}
  \psnote{the document has never (as far as I noticed) actually explained what a captable \emph{is}. probably that should have been much earlier}
  \jhbnote{I'm actually tempted to stop calling captables their own
    name and just treat them as GOTs and that a GOT can contain
    entries of various types: integer addresses, integer indices
    (think TLS DSO index), and capabilities.  The compiler/linker are
    responsible for avoiding type confusion by ensuring all uses and
    relocations for a given GOT entry use the same type.  I think on
    CHERI-MIPS at least we don't yet have a merged GOT and actually
    have separate sections.  I'm not sure it is required to have
    separate sections on other architectures which use .pltrel and
    jump slot relocations.}

\end{description}

\section{Types, casts, and addresses taken}

\psnote{somewhere this document should explain the (library or built-in) operations that one can do in hybrid on capabilities, what arithmetic one can do on them, what comparison does, etc., including concrete syntax (either general or by example)}

The C and \cpp{} languages are generally unchanged by the introduction of
\hybridCOrCpp{} support, except where specifically impacted by the use or
implementation of the capability pointer type.
Capability pointers will generally follow the rules, alignment requirements,
integer conversions, provenance, and so on laid out in the \textit{CHERI C/C++
Programming Guide}~\cite{UCAM-CL-TR-947}.
However, substantially different choices are made regarding integer types,
integer-pointer casts, and the address-taken operator, as described in this section.

\subsection{C-language types}

In \hybridCOrCpp{}, all existing language types retain their current uses in
order to maximize compatibility, including \cintptrt, \cuintptrt, and
\maxalignt.
This is inconvenient with respect to capability pointer types, which cannot
be stored (without a downcast to an integer pointer type, losing capability
metadata) in \cintptrt.
New types, \cintcapt and \cuintcapt, have been introduced, which are able to
hold capability pointer types -- but will be used only by source code that is
capability aware.

\subsection{Casts from integer pointers to capability pointers}

Casts from integer pointers to capability pointers will silently \rwnote{is
this still true?} generate a capability pointer value derived from the
Default Data Capability (DDC).
As a result, they will generally have highly permissive bounds and
permissions, which should be refined for the capability to offer spatial,
and not just referential, protection.
\psnote{I'm guessing the reader will have no idea what ``referential protection'' means...}

For example, if an integer pointer to a heap allocation is to be exposed to
co-process IPC using a capability, the code performing a cast to a capability
pointer will typically also need to set bounds and permissions suitably.
This use case may also require stronger alignment for the heap allocation than
the existing heap allocator provides, as the allocator will be unaware of
capability bounds precision requirements.
\psnote{give example code showing how this setting of bounds and permissions can be done?}

Because of these risks, CHERI Clang/LLVM will always generate a warning in
the absence of use of the new \ccode{\_\_cheri\_addr} qualifier on the target
type of a cast.
\bdnote{I think this is supposed to be \ccode{\_\_cheri\_tocap}. \ccode{\_\_cheri\_addr} is for capability to non-pointer integrer cases.}

\psnote{introducing \ccode{\_\_cheri\_addr} in passing -- better to be more explicit?}

\rwnote{Give examples of warnings?}

\nwfnote{Recommend \ccode{cheri\_address\_set}?}

\bdnote{We rarely use \_\_cheri\_tocap and instead use macros like \ccode{\_\_USER\_CAP} that perform some validation.  In the case of \ccode{\_\_USER\_CAP} we have a \ccode{\_\_USER\_CFROMPTR} that derives
from the process's DDC and (critically) ensures that sentinel values near (signed) zero are NULL-derived. We probably need to mention this here
or above when we talk about syscall arguments.}

\jhbnote{I think the takeaway about \ccode{\_\_USER\_CAP} is that it
  is not using language features (i.e. casts as being described here),
  but instead is a separate DSL built on top of C for use in the
  kernel.  I do think it is very much worth calling out that in the
  kernel, using DDC to derive caps for user addresses is almost always
  wrong, and that you want to be using the user's DDC explicitly
  instead.  For the purposes of this section about casts, simply
  talking about \ccode{\_\_cheri\_tocap} is ok.  However, we are missing an
  entire subsection here for ``Casts from capability to integer address``
  which needs to talk about \ccode{\_\_cheri\_addr}.}
 
\subsection{Casts from capability pointers to integer pointers}

\psnote{terminology: it's pretty easy to misread ``integer pointer'' as ``pointer to integer'', as that's normal C usage.  Maybe we can find a different term?  ``machine-word pointer'' ??}

Casts from capability pointers to integer pointers will take on a value that
is the integer difference between the address of the capability pointer and
the address of DDC.
These casts likewise discard capability metadata, preventing not only spatial
protection, but also referential protection.

Because of these risks, CHERI Clang/LLVM will always generate an error in
the absence of use of the new \ccode{\_\_cheri\_fromcap} qualifier on the
target type of a cast.

\rwnote{Say something about the associated built-ins?}

\rwnote{Give examples of warnings?}

\dcnote{We used to have instructions for this.  The goal for systems without
that was cap -> int pointer casts would do a CTestSubset, CSub, CCMOV sequence,
so casting from a not-in-DDC cap to an integer pointer would give either a
valid pointer or null.  The programmer invariant is that either the resulting
pointer can be used to access the entire object, or it is a detectable null.
The LLVM IR language reference now include this guarantee for address space
casts.}

\bdnote{In the CheriBSD kernel \ccode{\_\_cheri\_fromcap} is commonly used
for kernel-addresses that happen to be stored in capabilities,
but probably should be replaced by a macro that ensures that this is true
on systems where DDC includes userspace addresses (all of them today).}

\subsection{Address-of operator (\&)}

The \ccode{\&} operator takes the address of a C or \cpp{} variable, and
returns a pointer of a corresponding type.
We consider three approaches:

\begin{description}
\item[Integer pointer interpretation] is the approach taken by CHERI
  Clang/LLVM.
  In this approach, \ccode{\&} returns an integer pointer type, the default
  pointer type in \hybridCOrCpp{}.
  If then assigned to a capability pointer, the resulting bounds and
  permissions are derived from the Default Data Capability (DDC).

\nwfnote{Does that mean that I can't write something like...  \ccode{struct \{
int x; int y; \} * \_\_capability sp; int * \_\_capability xp = \&sp->x;} and
have it work unless \ccode{sp} is a subset of DDC?  I'd sort of expect \& to be
polymorphic in its argument, even if it isn't polymorphic in its return type as
given next.}
\bdnote{\&<something of capability type> remains a capability. This requires clarification.}

\item[Polymorphic \ccode{\&}] \morellovariant{is the approach taken by Morello
  Clang/LLVM.
  In this approach, \ccode{\&} is polymorphic: if assigned to a capability
  pointer, the resulting bounds and permissions are those of the underlying
  type.}
  
  \rwnote{Or are they of the underlying storage?}
\psnote{uneasy with the idea that ``the underlying type'' has unambiguous bounds and permissions... esp. given C's handling of arrays.  Spell out what this means?}
  
  \dcnote{Is this equivalent to making it return a capability and silencing
    warnings if you cast that value to an integer pointer without storing it?}

\item[Qualified \ccode{\&}]
  \futurevariant{This approach would qualify or modify the \ccode{\&} operator
  itself, perhaps in the form \ccode{\_\_capability \&}.
  This approach would avoid polymorphic behavior with the effect of the
  \ccode{\&} operator having different behavior based on others of a complex
  expression, or drawn from the type of the left-hand side of an assignment.
  Currently, no similar constructs to qualify or modify operators exist in the C
  language.}
\end{description}

Another potential approach, avoided in our work to date, is to support a
silent downcast from a capability-pointer typed \ccode{\&} operator.
In general, we feel strongly that casts from capability-pointer types to
integer-pointer types should generate warnings (or even errors) due to the
potential for lost protection information and confusing outcomes.
In large existing corpora of \hybridCOrCpp{} code, large numbers of warnings
would be produced
\psnote{by what exactly? by having a silent downcast??}
, and likely substantially distract from the goal of
producing safe code.

Regardless of the approach taken, it is important that the compiler warn if
there is a moderate likelihood that an assignment into a capability pointer
might contain default or too-broad bounds compared to the expectation of the
programmer.
Here, both the current CHERI Clang/LLVM and potential future CHERI Clang/LLVM
approaches require explicit warnings be generated, whereas the Morello
approach should avoid surprise at the cost of potentially problematic C
behavior.
\psnote{That last sentence is structured as an opposition -- but warnings and avoiding surprise seem consonant...}

\dcnote{This section is missing any description of C++ references.  I recall
that there were some fun corner cases here, for example if you have a
cap-pointer to an object (which may not be in DDC), what is the type of a
reference to that reference?}

\section{Application Programming Interfaces (APIs)}

\hybridCOrCpp{} does not mandate the introduction of new APIs beyond the
capability inspection and manipulation APIs documented in the
\textit{\purecapCOrCpp{} Programming Guide}~\cite{UCAM-CL-TR-947}.
However, C runtime environments may choose to provide extended versions of
existing library and system-call interfaces to support capability-pointer
arguments and return values.
Typically these will share the name of an existing API call, but append the
\ccode{\_c} suffix.
For the time being, we define extended interfaces to the heap allocator.

\subsection{C heap allocation}

\begin{description}
\item[\ccode{malloc\_c()}] Based on the \ccode{malloc()} API, this
  capability-enabled variant returns a capability pointer rather than integer
  pointer to a new suitably aligned and padded heap memory allocation.

\item[\ccode{free\_c()}] Based on the \ccode{free()} API, free a heap
  allocation, taking a capability pointer rather than an integer pointer
  argument.
\end{description}

Whether \ccode{free\_c()} supports freeing any heap allocations provided by
the un-extended \ccode{malloc()} interface will depend on the implementation,
and it is recommended that portable code only pass capability values returned
by \ccode{malloc\_c()} be passed to \ccode{free\_c()}.
This permits use of a different heap allocator implementation better able to
implement alignment or padding requirements, or, for example, able to support
temporal safety via capability revocation.

\subsection{POSIX memory mappings}

In the future, extensions to the POSIX memory-mapping API will be desirable to
support use cases such as capability-extended IPC between co-processes (see
Section~\ref{section:co-processes}).
However, currently POSIX APIs for memory mapping often have design mismatches
with CHERI requirements.

For example, capability bounds imprecision may mean that the memory mapping
created by \ccode{mmap()} is longer than the requested length -- but there is
no way to return that new length.
This is a particular problem because, unlike \ccode{free()}, \ccode{munmap()}
accepts a length.
problem for the \ccode{munmap()} API, which requires passing a length.
Similarly, for the purposes of temporal safety, non-reuse of portions of the
address space must be ensured globally until revocation has taken place, which
is not a use case well supported by the current API.

While incremental changes to introduce minor revisions to APIs, such as adding
a \ccode{mmap\_c()} are possible, it will be preferable to add new APIs better
reflecting CHERI's requirements rather than, for example, leak portions of
address space.

\section{Code examples}
\label{section:hybrid-c-examples}

In this section, we consider several current uses of \hybridCOrCpp{}.

\subsection{Hybrid CheriBSD kernel}

CheriBSD is an adaptation of the open-source FreeBSD operating system to
tightly incorporate CHERI support for memory protection and software
compartmentalization.
CheriBSD implements two kernel variants: a \hybridCOrCpp{} compilation, and a
\purecapCOrCpp{} variant.
The former, \hybridCOrCpp{} compilation, aims to support a CHERI-enabled userspace
without substantive changes to kernel-internal protection.
The latter additionally aims to use CHERI memory protection within the
kernel.
\psnote{say explicitly that we don't discuss the latter further here?  Otherwise it's confusing to set up the opposition but then only talk about the former below}
Both kernel variants support hybrid and pure-capability (CheriABI) userspace
process environments.

The \hybridCOrCpp{}-compiled CheriBSD kernel is almost entirely implemented
in conventional \COrCpp{} using integer pointers, except for its handling of
pointers to userspace (e.g., system-call arguments) and capability state,
which are manually annotated.
Capability pointers are used in this version of the kernel for three reasons:

\begin{enumerate}
\item to manage the dynamic state of capability-enabled kernel and user
  threads, as part of context management and exception handling;

\item in the implementation of tag-enabled virtual-memory abstractions, such
  as copy-on-write propagating tag bits; and

\item to handle pointer-type userspace system-call arguments as capabilities
  rather than integer pointers, both to enable capability-based behaviors
  (such as having opaque pointer types used in POSIX asynchronous I/O be able
  to hold capabilities), and to prevent confused deputy attacks (such as might
  occur if an omnipotent kernel ignored userspace-originated bounds on stack
  or heap allocations).
  \jhbnote{I think system-call argument is perhaps too limiting.  We
    also use capabilities for things like the initial stack layout
    during exec which is fabricated by the kernel and not provided as
    a syscall argument.  I think it might be best to say something
    like ``userspace pointers'' most of which come from
    user-originated system call arguments, but may also be
    instantiated by the kernel.}
\end{enumerate}

In order to unify system-call handling, user pointers originating in hybrid
ABI processes are converted into capability pointers at the system-call
boundary.%
\footnote{System-call arguments may point to structures themselves containing
  pointers, which the kernel must also translate.
  Such translation may occur at the syscall-boundry (e.g., the \ccode{struct
  iovec} in \ccode{readv}) or deep within the kernel (e.g., a driver-specific
  \ccode{ioctl} struct).}
%
This leads to coarse-grained bounds, but, for example, permits user pointers to
be used only in accessing userspace memory.
For pure-capability (CheriABI) processes, user pointer types originate as
capabilities, and are simply propagated as necessary.
This approach is moderately disruptive of the kernel source code: while there
are few structural changes, user-originated pointers are manually annotated
with \uucap{} throughout.

\subsection{The CheriBSD hybrid process environment}

The CheriBSD hybrid process environment modestly extends the existing native
non-CHERI process ABI on each architecture.
Capability registers and tagged memory are maintained, and signal-frame state
is extended, but initial process state, system-call arguments, and system-call
return values hold integer pointers rather than capability pointers.

In the usermode portion of the CheriBSD hybrid ABI process environment, some
system libraries are extended with modest capability annotations to support
capability use.
For example, \ccode{memcpy()} and \ccode{sort()} must propagate tags rather
than perform byte-wise copies, in order to preserve pointers embedded in data
structures across memory copies.
In addition, new variants of a few functions are added which use
capability pointers rather than integer pointers such as
\ccode{memcpy\_c()} and \ccode{memset\_c()}.
\psnote{Apart from that??}
\psnote{[limited/weak] ?}
However, support for capabilities is generally weak within the runtime.
As examples, we have not implemented a \ccode{malloc\_c()} variant that would
return a capability while accounting for alignment, padding, and
capability bounds imprecision, nor have we endeavoured to implement temporal memory safety.
\psnote{hard to parse the latter parts of the preceding sentence}
\rwnote{Possibly now improved?}
\jhbnote{I think it is confusing to mention \ccode{malloc\_c()} this
  early as the idea of \ccode{*\_c()} variants isn't yet explained.  I
  have tried to rectify this a bit.}
System calls are also not generally extended to support capability arguments
or return values, limiting the useful origins for capabilities other than as
derived from the Default Data Capability (DDC) or Program Counter Capability
(PCC).
For example, to support non-DDC-derived memory mappings, new \ccode{mmap\_c}
and \ccode{munmap\_c} system-call variants would be desirable.

\subsection{Co-process IPC libraries in hybrid processes}
\label{section:co-processes}

We have been developing an experimental CHERI-based compartmentalization model
called \textit{co-located processes} (or \textit{co-processes} for short) on
the CheriBSD operating system.
In this model, a set of UNIX processes are colocated within a single shared
virtual address space, but kept separate by virtue of CHERI capabilities:
processes only ever receive capabilities to their own individual mappings.

Co-process Inter-Process Communication (IPC) relies on bridging those
processes by allowing a limited set of capabilities to be shared between those
processes.
In this approach, hybrid capability-extended IPC libraries would have access
to the memory of remote processes using CHERI capabilities, but the majority
of code in a hybrid process would not have that direct access.
Depending on performance and compatibility objectives, knowledge of
co-processes could be restricted to low-level IPC libraries, or be propagated
higher in the IPC stack, reducing the need for memory copying.

This is an area where our research is ongoing, but access to high-performance
CHERI IPC with only limited recompilation and extension to existing software
packages is a potentially promising area.

\subsection{Hybrid language runtimes}

Managed language runtimes are a point of particular performance concern with
CHERI, as they often have a high density of pointers in their dynamic memory
access patterns.
This makes them potentially more sensitive to CHERI's pointer-size growth in
pure-capability compilation.
On the other hand, CHERI capabilities offer the opportunity to improve
robustness and performance by allowing selective use of capabilities to
protect key types and provide hardware assistance for bounds checking.
On the whole, our recommendation has been to consider compiling language
runtimes as pure-capability code making selective use of integer pointers
(e.g., into language-specific heaps reached via capabilities), but
another approach would be to compile them as hybrid code making selective use
of capabilities -- e.g., when accessing bounded arrays in the heap.
\dcnote{It feels quite dangerous to have a single sentence about the
recommended way of doing things and a long document about the not-recommended
way.}
\nwfnote{Even if we've cited it recently, this is probably a good time to
emphatically point people at the right way~\cite{UCAM-CL-TR-947}.}


\section{Limitations}

We consider two general sources of limitations, which are inevitably
intertwined:

\begin{itemize}
\item An uncomfortable programming model requiring manual annotation of
  source code
\item Incomplete referential, spatial, and temporal safety
\end{itemize}

\subsection{Programming model}

The \hybridCOrCpp{} programming model is a challenging one requiring
considerable care.
In general, it is our recommendation that the function of \hybridCOrCpp{} is
not improved memory safety, but rather interoperability with more
capability-centered \purecapCOrCpp{} across a fairly hard ABI boundary.
For example, the hybrid CheriBSD kernel variant implements its CheriABI
system-call layer in \hybridCOrCpp{} so that it can interact with capabilities
going to and from pure-capability user processes -- while itself primarily
relying on integer pointers internally.

In C code written to employ \ccode{typedef}s for pointer types, the level
of disruption associated with \uucap{} may not be enormous.
However, the useful setting of bounds is key.
In the use cases we imagine, such as kernel ABI interfaces and IPC libraries,
capabilities are used only in situations in which suitable bounds and
permissions are natural products of the code implementing those services.
For example, if capabilities refer to a shared IPC memory object outside the
DDC-addressable memory of \hybridCOrCpp{} code, then the IPC library has
gained access to that capability from the party that allocated it in a
capability-aware way.
Similarly, hybrid kernels exchanging capabilities with user code will ensure
that suitable bounds are set of capabilities transmitted to userlevel, and
they will also return to the kernel with those bounds (or narrower ones)
intact.

Were more general-purpose programming the aim, a set of extended
capability-aware APIs (e.g., \ccode{malloc\_c()}, \ccode{mmap\_c()}, and so
on), returning and managing capability properties explicitly through
\uucap{}-extended types, would be required.

\jhbnote{Somewhere we should perhaps point out that existing hybrid
  ABIs don't try to ``respect'' DDC.  In particular, as far as I am
  aware, \ccode{mmap()} will happily return integer pointers that are
  out of bounds of DDC if userland has constrained its DDC.  I think
  the expectation is that if hybrid userland wants to modify DDC, it
  has to wrap system calls, etc.}

\subsection{Incomplete referential, spatial, and temporal safety}

\hybridCOrCpp{} is a relatively weak integration of CHERI support with
the language and run-time environment offering limited memory protection only
in very specific circumstances.
Except where Application Programming Interfaces (APIs) have been extended to
support explicit capability arguments or return values, pointers will not
be protected by capability integrity, pointer provenance validity, bounds,
permissions, or monotonicity.
For example, \cfunc{malloc} will not return pointers as capabilities with
bounds set, and also may not align or pad allocations such that they can be
bounded imprecisely without loss of spatial protection relative to adjacent
allocations.
Further, the control-flow and other data structures used by C and the C
runtime themselves do not use capabilities, and so some of the resilience to
exploitation found in \purecapCOrCpp{}, such as return addresses and stack
pointers implemented using capabilities, are not found in \hybridCOrCpp{}.
Specific limitations by protection type are:

\begin{itemize}
\item Integer pointers, including those returned by default by various memory allocators, do
  not implement capability protections including tag, bounds, and permissions.
  As a result, integer pointers suffer from a lack of provenance validity and
  spatial safety.
  \bdnote{Furthermore, an unmodified allocator may produce allocations lacking
  sufficient alignment support precise bounds.}

\item Whenever a capability pointer type is cast to an integer pointer type,
  its capability metadata, including tag, bounds, and permissions, are lost.
  The resulting integer pointer will be dereferenced relative to the Default
  Data Capability (DDC), and uses those bounds and permissions, which may be
  substantial, and provenance validity is not implemented.

  \dcnote{And, more importantly, casts are not symmetric.  Int to cap pointer
    casts always allow you to access everything with the result that you could
    access with the source.  Caps to int casts may give null.  In to cap to int
    casts are always safe round trips, cap to int to cap may give either a
    capability with more rights or the null capability.}

  \nwfnote{Casts to integers of tag-clear capabilities trap or succeed?}

\item Whenever an integer pointer type is cast to a capability pointer type,
  the compiler will derive the new capability from DDC without refining its
  bounds or permissions, offering provenance validity but not privilege
  minimization.

\item Both explicit function pointers and implied pointers such as return
  addresses and PLT entries are implemented as integer pointers, and therefore
  lack provenance validity, bounds, and permissions.
  Explicit capability pointers will have greater protection, but only subject
  to suitable setting of their metadata.

\item Other data structures implemented by the language runtime and compiler,
  such as the stack, are also implemented using integer pointers that do not
  implement provenance validity, bounds, or permissions.

\item CHERI temporal safety relies on capability implementation of pointers,
  so that the tag, bounds, and permissions can be utilized to detect stale
  pointers to quarantined memory.
  As many pointers in a hybrid program are implemented using integers,
  temporal safety is unable to precisely find all pointers or arrange for
  their suitably atomic replacement.
  \dcnote{I think I'd phrase this slightly differently.  From the perspective
    of revocation, DDC (which encompases the C heap, stack, and image), is a
    single object and so no object reachable from the hybrid heap / stack /
    globals will ever be revoked.}
  \nwfnote{Well, a capability held within DDC that points to a revokable IPC
    object (i.e., an object not within DDC) is still subject to revocation
    as per usual.  I'd phrase it as ``subobjects of DDC'' will not be revoked
    soundly even if revocation is attempted.  In a sense DDC and the
    VMMAP-bearing caps handed to malloc are analogous here: malloc would be
    free (hah) to reconstruct a pointer to a free/revoked region incorrectly,
    but we (endeavour to) ensure that it doesn't.}
\end{itemize}

As a result of these limitations, privilege minimization in the language
runtime, control-flow mechanisms, and user data types is largely unimplemented.
Protections are only refined and enforced for programmer-selected types, which
can offer substantial value in specific scenarios, but is not a source of
general robustness for the C and \cpp{} implementations.
All privilege reductions are programmer-implemented through explicit calls to
narrow bounds and permissions.

\subsection{Origins of bounds and permissions}

Pointers implemented by capabilities by definition carry capability metadata:
tag, bounds, permissions, and so on.
In general, the \hybridCOrCpp{} runtime has much more limited information to
use as inputs for this metadata.
\psnote{...than pure-capability CHERI C/C++ (hmm -- maybe it's worth still keeping the ``pure'' terminology for this document, to make such oppositions clear?)}
Where the address-of operator is used with a capability modifier, linkage
information (such as the captable) can be used as an origin of bounds for
global variables, and the stack allocator itself can provide bounds for local
variables.
Heap allocation requires API extensions that are not currently present (see
above).

\section{Further reading}
\label{sec:further_reading}

The primary reference for the CHERI Instruction-Set Architecture (ISA) is the
ISA specification; at the time of writing, the most recent version is CHERI
ISAv8~\cite{UCAM-CL-TR-951}:

\smallskip
\noindent
\url{https://www.cl.cam.ac.uk/techreports/UCAM-CL-TR-951.pdf}
\smallskip

\noindent
Our technical report, \textit{\citefield{UCAM-CL-TR-941}{title}}, provides a high-level
overview of the CHERI architecture, ISA modeling, hardware implementations,
and software stack~\cite{UCAM-CL-TR-941}:

\smallskip
\noindent
\url{https://www.cl.cam.ac.uk/techreports/UCAM-CL-TR-941.pdf}
\smallskip

\noindent
The primary reference for \purecapCOrCpp{} is our technical report,
\textit{\citefield{UCAM-CL-TR-947}{title}}~\cite{UCAM-CL-TR-947}:

\smallskip
\noindent
\url{https://www.cl.cam.ac.uk/techreports/UCAM-CL-TR-947.pdf}
\smallskip

\noindent
We published a paper on idiomatic C and spatial memory protection at ASPLOS
2015~\cite{ChisnallCPDP11}:

\smallskip
\noindent
\url{https://www.cl.cam.ac.uk/research/security/ctsrd/pdfs/201503-asplos2015-cheri-cmachine.pdf}
\smallskip

\noindent
We published a paper on CheriABI and the adaptation of a complete OS userspace
and application suite to a pure-capability process environment at ASPLOS
2019~\cite{davis2019:cheriabi}:

\smallskip
\noindent
\url{https://www.cl.cam.ac.uk/research/security/ctsrd/pdfs/201904-asplos-cheriabi.pdf}
\smallskip

\noindent
We also released an extended technical-report version of this paper that
includes greater implementation detail~\cite{UCAM-CL-TR-932}:

\smallskip
\noindent
\url{https://www.cl.cam.ac.uk/techreports/UCAM-CL-TR-932.pdf}
\smallskip

\section{Acknowledgements}

\rwnote{Update feedback acknowledgments here.}

We gratefully acknowledge the helpful feedback from our colleagues, including
Ruben Ayrapetyan, Silviu Baranga, and Kevin Brodsky.
This work was supported by the Defense Advanced Research Projects Agency (DARPA) and the Air Force Research Laboratory (AFRL), under contracts
FA8750-10-C-0237 (``CTSRD'') and HR0011-18-C-0016 (``ECATS'').
The views, opinions, and/or findings contained in this report are those of the authors and should not be interpreted as representing the official views or policies of the Department of Defense or the U.S. Government.
This work was supported in part by the Innovate UK project Digital Security by
Design (DSbD) Technology Platform Prototype, 105694.
This project has received funding from the European Research Council
(ERC) under the European Union's Horizon 2020 research and innovation programme (grant agreement No 789108), ERC Advanced Grant ELVER.
We also acknowledge the EPSRC REMS Programme Grant (EP/K008528/1), Arm Limited,
HP Enterprise, and Google, Inc.
Approved for Public Release, Distribution Unlimited.

\printbibliography

\end{document}
